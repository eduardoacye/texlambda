
%%% Local Variables:
%%% mode: latex
%%% TeX-master: t
%%% End:

\documentclass[12pt]{article}

\usepackage[margin=1.5cm]{geometry}
\usepackage[spanish]{babel}
\usepackage[utf8]{inputenc}
\usepackage{texlambda}
%\usepackage{concmath}

\begin{document}

\begin{center}
  {\Huge \TeXLaMbDa}
\end{center}

\subsection*{Introducción}
Este paquete permite escribir en el modo de matemáticas de \LaTeX\ términos bien formados del cálculo lambda sin tipos.\\

Los comandos principales son \texttt{\textbackslash lc\{x\}} y \texttt{\textbackslash lc*\{x\}}. El primero traduce el término lambda \texttt{x} en código para el modo matemáticas de \LaTeX\ sin abusos de notación, el segundo traduce a \texttt{x} de la manera mas concisa posible, utilizando los abusos de notación estándar en la literatura del cálculo lambda. En ambos casos, \texttt{x} puede ser escrito con o sin abuso de notación o una mescolanza de ambos estilos.\\

La sintáxis aceptada es:
\begin{itemize}
\item[\S] Si \texttt{x} es una secuencia de caracteres sin espacios, entonces \texttt{x} es un término lambda aceptado (\emph{átomo}).

\item[\S] Si \texttt{x} y \texttt{y} son términos lambda, entonces \texttt{(x y)} es un término lambda aceptado (\emph{aplicación}).

\item[\S] Si \texttt{x} es una secuencia de caracteres sin espacios y \texttt{y} es un término lambda aceptado, entonces \texttt{(\textbackslash x.y)} es un término lambda aceptado (\emph{abstracción}).
\end{itemize}

Los abusos de notación son:
\begin{itemize}
\item[\S] Si \texttt{x} es una aplicación o una abstracción, se pueden ignorar los paréntesis.
\item[\S] Si \texttt{x} es una abstracción cuyo cuerpo es otra abstracción, se pueden agrupar los argumentos de ambas abstracciones, e.g \texttt{(\textbackslash x.(\textbackslash y.M))} es equivalente a \texttt{(\textbackslash x y.M)}.
\item[\S] Si \texttt{x} es una aplicación anidada con asociación a la izquierda, se pueden escribir los términos en las aplicaciones de manera consecutiva, e.g \texttt{(((a b)c)d)} es equivalente a \texttt{(a b c d)}.
\end{itemize}

\subsection*{Ejemplos}

\subsubsection*{Átomos}
Escribiendo \texttt{\textbackslash lc\{x\}} se obtiene \( \lc{x} \).

Escribiendo \texttt{\textbackslash lc*\{x\}} se obtiene \( \lc*{x} \).


\subsubsection*{Abstracción lambda}
Escribiendo \texttt{\textbackslash lc\{\textbackslash x.x\}} se obtiene \( \lc{\x.x} \).

Escribiendo \texttt{\textbackslash lc*\{\textbackslash x.x\}} se obtiene \( \lc*{\x.x} \).

\subsubsection*{Aplicación lambda}
Escribiendo \texttt{\textbackslash lc\{x y z\}} se obtiene \( \lc{x y z} \).

Escribiendo \texttt{\textbackslash lc*\{x y z\}} se obtiene \( \lc*{x y z} \).

\subsubsection*{Numerales de Church}
Escribiendo \texttt{\textbackslash lc\{\textbackslash f x.f(f(f(f(f x))))\}} se obtiene \( \lc{\f x.f(f(f(f(f x))))} \).

Escribiendo \texttt{\textbackslash lc*\{\textbackslash f x.f(f(f(f(f x))))\}} se obtiene \( \lc*{\f x.f(f(f(f(f x))))} \).

\subsubsection*{Términos lambda variados}
Escribiendo \texttt{\textbackslash lc\{x y z (y x)\}} se obtiene \( \lc{x y z (y x)} \).

Escribiendo \texttt{\textbackslash lc*\{(((x y) z) (y x))\}} se obtiene \( \lc*{(((x y) z) (y x))} \).

\bigskip

Escribiendo \texttt{\textbackslash lc\{\textbackslash x.u x y\}} se obtiene \( \lc{\x.u x y} \).

Escribiendo \texttt{\textbackslash lc*\{(\textbackslash x.((u x) y))\}} se obtiene \( \lc*{(\x. ((u x) y))} \).

\bigskip

Escribiendo \texttt{\textbackslash lc\{\textbackslash u.u(\textbackslash x.y)\}} se obtiene \( \lc{\u.u(\x.y)} \).

Escribiendo \texttt{\textbackslash lc*\{(\textbackslash u.(u (\textbackslash x.y)))\}} se obtiene \( \lc*{(\ u. (u (\x. y)))} \).

\bigskip

Escribiendo \texttt{\textbackslash lc\{(\textbackslash u.v u u)z y\}} se obtiene \( \lc{(\u.v u u)z y} \).

Escribiendo \texttt{\textbackslash lc*\{(((\textbackslash u.((v u) u)) z) y)\}} se obtiene \( \lc*{(((\ u. ((v u) u)) z) y)} \).

\bigskip

Escribiendo \texttt{\textbackslash lc\{u x(y z)(\textbackslash v.v y)\}} se obtiene \( \lc{u x(y z)(\v.v y)} \).

Escribiendo \texttt{\textbackslash lc*\{(((u x)(y z))(\textbackslash v.(v y)))\}} se obtiene \( \lc*{(((u x) (y z)) (\v. (v y)))} \).

\bigskip

Escribiendo \texttt{\textbackslash lc\{(\textbackslash x y z.x z(y z))u v w\}} se obtiene \( \lc{(\x y z. x z(y z)) u v w} \).

Escribiendo \texttt{\textbackslash lc*\{((((\textbackslash x.(\textbackslash y.(\textbackslash z.((x z)(y z)))))u)v)w)\}} se obtiene \( \lc*{((((\x. (\y. (\ z. ((x z) (y z))))) u) v) w)} \).

\subsection*{Estilos}

Para obtener diferentes estilos de términos, se puede utilizar el comando \texttt{\textbackslash lc\{x\}} con argumentos extras: \texttt{\textbackslash lc[args]\{x\}}, donde \texttt{x} es un término lambda como en los anteriores comandos y \texttt{args} son las banderas (o \emph{flags}) que determinan el formato del término.\\

Las banderas admitidas son \texttt{s}, \texttt{v}, \texttt{l}, \texttt{d} y \texttt{p}. Si ejecutas el comando \texttt{./texlambda --help} obtendrás la siguiente descripción de las banderas:

\begin{verbatim}
TeX-LaMbDa [ <option> ... ] <str>
 where <option> is one of
  -s, --spaced : Spaced terms mode - Introduces spacing
  -v, --bold-variables : Bold variables mode - Make variable names bold
  -l, --bold-lambdas : Bold lambdas mode - Makes lambdas bold
  -d, --bold-dots : Bold dots mode - Makes dots bold
  -p, --bold-parentheses : Bold parentheses mode - Makes parentheses bold
  -e, --explicit : Explicit mode - Removes abuse of notation
  --help, -h : Show this help
  -- : Do not treat any remaining argument as a switch (at this level)
 Multiple single-letter switches can be combined after one `-'; for
  example: `-h-' is the same as `-h --'
\end{verbatim}

El modo explícito es controlado por el modificador estrella en el comando \texttt{lc}, así que no debes utilizar la bandera \texttt{e}.\\

\subsection*{Ejemplos de modificación de estilos}
Por ejemplo, para obtener ``\emph{negritas}'' en las lambdas y puntos, se utiliza el comando \texttt{\textbackslash lc[ld]\{x\}}, también sirve usar como banderas \texttt{dl} ya que el orden no importa:

\[ \lc{(\x y z. x z(y z)) u v w} \]\\

Si queremos tener únicamente los átomos en ``\emph{negritas}'' se escribe \texttt{\textbackslash lc[v]\{x\}}:

\[ \lc{(\x y z. x z(y z)) u v w} \]\\

La versión no explícita de este término sería \texttt{\textbackslash lc*[v]\{x\}}:

\[ \lc*{(\x y z. x z(y z)) u v w} \]\\

Y si deseamos un término lambda mas espaciado se puede utilizar \texttt{\textbackslash lc*[vs]\{x\}}:

\[ \lc*{(\x y z. x z(y z)) u v w} \]\\

Si queremos tener todo en ``\emph{negritas}'' excepto las variables, utilizamos \texttt{\textbackslash lc[pdl]\{x\}}:

\[ \lc{(\x y z. x z(y z)) u v w} \]\\

Y con \texttt{\textbackslash lc*[pdl]\{x\}}:

\[ \lc*{(\x y z. x z(y z)) u v w} \]\\

Es posible asignar las banderas por defecto utilizando el comando \texttt{\textbackslash lcflags\{args\}}, de tal manera que si se asignan banderas utilizando este comando, todos los términos lambda escritos con \texttt{\textbackslash lc} o \texttt{\textbackslash lc*} sin argumentos extra, utilizarán estas banderas. Por ejemplo al escribir \texttt{\textbackslash lcflags\{pld\}}:
\lcflags{pdl}
\begin{itemize}
\item[\S] Escribiendo \texttt{\textbackslash lc\{\textbackslash f x.f(f(f(f x)))\}} se obtiene \( \lc{\f x.f(f(f(f x)))} \)
\item[\S] Escribiendo \texttt{\textbackslash lc*\{\textbackslash f x.f(f(f(f x)))\}} se obtiene \( \lc*{\f x.f(f(f(f x)))} \)
\end{itemize}
\lcflags{}

Con este comando se pueden redefinir las banderas y escribiendo \texttt{\textbackslash lcflags\{\}} se eliminan, regresando a los valores por defecto originales.\\

Por el momento es lo único que puede estilizar el programa \texttt{texlambda}, sin embargo estoy trabajando en poder realizar con comandos de latex y de manera declarativa \betaredu\ y \alphaconv.\\

\subsection*{Operaciones y equivalencias}

El paquete \TeXLaMbDa también tiene algunos comandos para escribir operaciones y equivalencias utilizadas frecuentemente en la literatura. Por el momento son:

\begin{itemize}
\item[\S] \texttt{\textbackslash betaredu} $\rightarrow$ \betaredu.
\item[\S] \texttt{\textbackslash alphaconv} $\rightarrow$ \alphaconv.
\item[\S] \texttt{\$\textbackslash synteq\$} $\rightarrow$ $\synteq$.
\item[\S] \texttt{\$\textbackslash termlen\{\textbackslash lc\{\textbackslash x.x\}\}\$} $\rightarrow$ $\termlen{\lc{\x.x}}$.
\item[\S] \texttt{\textbackslash alphacong} $\rightarrow$ \alphacong.
\end{itemize}

Hacen falta muchas para completar las que utilizaré en mi tesis de licenciatura, sin embargo, aún no establezco una notación fija para las operaciones.

\end{document}
%%% Local Variables:
%%% mode: latex
%%% TeX-master: t
%%% End:
