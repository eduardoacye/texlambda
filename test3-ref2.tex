\subsection*{Ejemplos}

\subsubsection*{Átomos}
Escribiendo \texttt{\textbackslash lc\{x\}} se obtiene \( \lc{x} \).

Escribiendo \texttt{\textbackslash lc*\{x\}} se obtiene \( \lc*{x} \).


\subsubsection*{Abstracción lambda}
Escribiendo \texttt{\textbackslash lc\{\textbackslash x.x\}} se obtiene \( \lc{\x.x} \).

Escribiendo \texttt{\textbackslash lc*\{\textbackslash x.x\}} se obtiene \( \lc*{\x.x} \).

\subsubsection*{Aplicación lambda}
Escribiendo \texttt{\textbackslash lc\{x y z\}} se obtiene \( \lc{x y z} \).

Escribiendo \texttt{\textbackslash lc*\{x y z\}} se obtiene \( \lc*{x y z} \).

\subsubsection*{Numerales de Church}
Escribiendo \texttt{\textbackslash lc\{\textbackslash f x.f(f(f(f(f x))))\}} se obtiene \( \lc{\f x.f(f(f(f(f x))))} \).

Escribiendo \texttt{\textbackslash lc*\{\textbackslash f x.f(f(f(f(f x))))\}} se obtiene \( \lc*{\f x.f(f(f(f(f x))))} \).

\subsubsection*{Términos lambda variados}
Escribiendo \texttt{\textbackslash lc\{x y z (y x)\}} se obtiene \( \lc{x y z (y x)} \).

Escribiendo \texttt{\textbackslash lc*\{(((x y) z) (y x))\}} se obtiene \( \lc*{(((x y) z) (y x))} \).

\bigskip

Escribiendo \texttt{\textbackslash lc\{\textbackslash x.u x y\}} se obtiene \( \lc{\x.u x y} \).

Escribiendo \texttt{\textbackslash lc*\{(\textbackslash x.((u x) y))\}} se obtiene \( \lc*{(\x. ((u x) y))} \).

\bigskip

Escribiendo \texttt{\textbackslash lc\{\textbackslash u.u(\textbackslash x.y)\}} se obtiene \( \lc{\u.u(\x.y)} \).

Escribiendo \texttt{\textbackslash lc*\{(\textbackslash u.(u (\textbackslash x.y)))\}} se obtiene \( \lc*{(\ u. (u (\x. y)))} \).

\bigskip

Escribiendo \texttt{\textbackslash lc\{(\textbackslash u.v u u)z y\}} se obtiene \( \lc{(\u.v u u)z y} \).

Escribiendo \texttt{\textbackslash lc*\{(((\textbackslash u.((v u) u)) z) y)\}} se obtiene \( \lc*{(((\ u. ((v u) u)) z) y)} \).

\bigskip

Escribiendo \texttt{\textbackslash lc\{u x(y z)(\textbackslash v.v y)\}} se obtiene \( \lc{u x(y z)(\v.v y)} \).

Escribiendo \texttt{\textbackslash lc*\{(((u x)(y z))(\textbackslash v.(v y)))\}} se obtiene \( \lc*{(((u x) (y z)) (\v. (v y)))} \).

\bigskip

Escribiendo \texttt{\textbackslash lc\{(\textbackslash x y z.x z(y z))u v w\}} se obtiene \( \lc{(\x y z. x z(y z)) u v w} \).

Escribiendo \texttt{\textbackslash lc*\{((((\textbackslash x.(\textbackslash y.(\textbackslash z.((x z)(y z)))))u)v)w)\}} se obtiene \( \lc*{((((\x. (\y. (\ z. ((x z) (y z))))) u) v) w)} \).